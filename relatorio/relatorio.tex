\documentclass[conference]{IEEEtran}
\usepackage[portuguese]{babel} % vai trocar automaticamente, por exemplo, Table por tabela
\usepackage{cite}
\usepackage{amsmath,amssymb,amsfonts}
\usepackage{algorithmic}
\usepackage{graphicx}
\usepackage{textcomp}
\usepackage{xcolor}
\usepackage{pgfplots}			%Importação de imagens tikz, carregar depois de xcolor
\usetikzlibrary{plotmarks}
%\usetikzlibrary{external}
%\tikzexternalize[prefix=tikz/]
\pgfplotsset{compat=newest}
\pgfkeys{/pgf/number format/.cd,1000 sep={}}%para não colocar , nos tikz para separar milhar

\usepackage[nolist]{acronym}
\usepackage{multirow}
\usepackage{booktabs}
\usepackage{verbatim}

\def\BibTeX{{\rm B\kern-.05em{\sc i\kern-.025em b}\kern-.08em
    T\kern-.1667em\lower.7ex\hbox{E}\kern-.125emX}}
\begin{document}

\begin{acronym}[TDMA]
    \acro{IEEE}{\textit{Institute of Electrical and Electronics Engineers}}
    \acro{ABNT}{Associação Brasileira de Normas Técnicas}
\end{acronym}

\newlength\figureheight
\newlength\figurewidth


\title{Redes 1 - Trabalho 1}

\author{\IEEEauthorblockN{\textbf{Fabiano A. de Sá Filho}}
\IEEEauthorblockA{\textit{Departamento de Informática} \\
\textit{Universidade Federal do Paraná -- UFPR}\\
GRR20223831 \\
fabiano.filho@ufpr.br}
}

\maketitle

\begin{abstract}
Este trabalho teve como objetivo a modelagem de um problema de Programação Linear Real, bem como a implementação generalizada em C, a qual gera o programa linear para qualquer instância do problema. A saída pode então ser alimentada ao \textbf{\textit{lp\_solve}} para se obter a solução da instância dada. Neste relatório, descrevo a modelagem do problema e o programa linear resultante; os arquivos para a execução do programa em C seguem em anexo.
\end{abstract}




%\nocite{*}
%\bibliographystyle{IEEEtran}
%\bibliography{otimizacao}
\end{document}
